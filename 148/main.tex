\documentclass[a4paper, 12pt]{article}
\usepackage[utf8]{inputenc}
\usepackage[russian,english]{babel}
\usepackage[T2A]{fontenc}
\usepackage[left=10mm, top=20mm, right=10mm, bottom=15mm, footskip=13mm]{geometry}
\usepackage{indentfirst}
\usepackage{amsmath,amssymb}
\usepackage{graphicx}
\usepackage[italicdiff]{physics}
\usepackage{float}
\usepackage{array}
\usepackage{physics}
\graphicspath{ {shema/} {graphic/} }
\usepackage{caption}
\captionsetup[figure]{name=Рисунок}
  
\title{Отчет по лабораторной работе 1.4.8

ИЗМЕРЕНИЕ МОДУЛЯ ЮНГА
МЕТОДОМ АКУСТИЧЕСКОГО
РЕЗОНАНСА}

\author{Максим Осипов, Б03-504}
\date{05.11.2025}

\begin{document}
\maketitle

\section{Аннотация}

\textbf{Цель работы:} исследовать явление акустического резонанса в тонком стержне; измерить скорость распространения продольных звуковых колебаний в тонких стержнях из различных материалов и различных размеров; измерить модули Юнга различных материалов.

\textbf{В работе используются:} генератор звуковых частот, частотомер, осциллограф, электромагнитные излучатель и приёмник колебаний, набор стержней из различных материалов.

\section{Теоретическая справка}

\subsection{Основные характеристики упругих свойств}

Основной характеристикой упругих свойств твёрдого тела является его \emph{модуль Юнга} $E$. Согласно закону Гука, если к элементу среды приложено некоторое механическое напряжение $\sigma$, действующее вдоль некоторой оси $x$ (напряжения по другим осям при этом отсутствуют), то в этом элементе возникнет относительная деформация вдоль этой же оси $\varepsilon = \Delta x / x_0$, определяемая соотношением:
\begin{equation}
    \sigma = \varepsilon E
    \label{eq:1}
\end{equation}

Распространение акустических волн обеспечивается за счёт упругости и инерции среды. Волны сжатия/растяжения, распространяющиеся вдоль оси, по которой происходит деформация, называются \emph{продольными}. Скорость распространения продольной акустической волны в простейшем случае длинного тонкого стержня определяется соотношением:
\begin{equation}
    u = \sqrt{\frac{E}{\rho}}
    \label{eq:2}
\end{equation}
где $\rho$ — плотность среды.

\subsection{Собственные колебания стержня. Стоячие волны}

В случае гармонического возбуждения колебаний с частотой $f$ продольная волна в тонком стержне может быть представлена в виде суперпозиции двух бегущих навстречу гармонических волн:
\begin{equation}
    \xi(x, t) = A_1 \sin(\omega t - kx + \varphi_1) + A_2 \sin(\omega t + kx + \varphi_2)
    \label{eq:8}
\end{equation}
где $\omega = 2\pi f$ — циклическая частота, $k = 2\pi/\lambda$ — волновое число.

Для свободных (незакреплённых) концов стержня граничные условия:
\begin{equation}
    \sigma(0) = 0 \rightarrow \left. \frac{\partial \xi}{\partial x} \right|_{x=0} = 0; \quad \sigma(L) = 0 \rightarrow \left. \frac{\partial \xi}{\partial x} \right|_{x=L} = 0
    \label{eq:9}
\end{equation}

Из граничных условий следует, что амплитуды и фазы падающей и отражённой волн одинаковы:
\begin{equation}
    A_1 = A_2
    \label{eq:10}
\end{equation}
\begin{equation}
    \varphi_1 = \varphi_2
    \label{eq:11}
\end{equation}

С учётом этих условий функция (\ref{eq:8}) преобразуется к виду гармонической стоячей волны:
\begin{equation}
    \xi(x, t) = 2A \cos(kx) \sin(\omega t + \varphi)
    \label{eq:12}
\end{equation}

Второе граничное условие (\ref{eq:9}) приводит к уравнению $\sin kL = 0$, решения которого определяют набор допустимых значений волновых чисел:
\begin{equation}
    k_n L = \pi n, \quad n = 1, 2, 3, \dots
    \label{eq:13}
\end{equation}

Выражая через длину волны $\lambda = 2\pi/k$:
\begin{equation}
    \lambda_n = \frac{2L}{n}, \quad n \in \mathbb{N}
    \label{eq:13prime}
\end{equation}

Допустимые значения частот (собственные частоты колебаний):
\begin{equation}
    f_n = \frac{u}{\lambda_n} = n \frac{u}{2L}, \quad n \in \mathbb{N}
    \label{eq:14}
\end{equation}

Именно при совпадении внешней частоты с одной из частот $f_n$ в стержне возникает акустический резонанс.



\begin{figure}[h]
\centering
\includegraphics[width=1.0\textwidth]{2025-11-04 (12).png}
\caption{. Собственные продольные колебания стержня с незакреплёнными концами
(для наглядности изображение дано не в масштабе, реальные смещения малы
по сравнению с длиной стержня}
\label{fig1}
\end{figure}


Амплитуда колебаний смещения среды распределена вдоль стержня по гармоническому закону: $\xi_0(x) = 2A \cos kx$. Точки с максимальной амплитудой называются \emph{пучностями смещения}, точки с минимальной (нулевой) амплитудой — \emph{узлами смещения}. Согласно закону Гука (\ref{eq:1}) в пучности смещения имеет место узел напряжения, и, наоборот, в узлах смещения имеется пучность напряжения.

\newpage

\section{Экспериментальная установка}

\begin{figure}[h]
\centering
\includegraphics[width=1.0\textwidth]{рис 1.png}

\label{fig1}
\end{figure}

 Исследуемый
стержень 5 размещается на стойке 10. Возбуждение и приём колебаний в
стержне осуществляются электромагнитными преобразователями 4 и 6,
расположенными рядом с торцами стержня. Крепления 9, 11 электромагнитов дают возможность регулировать их расположение по высоте, а
также перемещать вправо-влево по столу 12.
Электромагнит 4 служит для возбуждения упругих механических продольных колебаний в стержне. На него с генератора звуковой частоты 1 подаётся сигнал синусоидальной формы: протекающий в катушке электромагнита ток создаёт пропорциональное ему магнитное поле, вызывающее периодическое воздействие заданной частоты на торец стержня (к торцам
стержней из немагнитных материалов прикреплены тонкие стальные
шайбы). Рядом с другим торцом стержня находится аналогичный электромагнитный датчик 6, который служит для преобразования механических
колебаний в электрические. Принцип работы электромагнитных датчиков
описан подробнее ниже.
Сигнал с выхода генератора поступает на частотомер 2 и на вход
канала X осциллографа 3. ЭДС, возбуждаемая в регистрирующем электромагните 6, пропорциональная амплитуде колебаний торца стержня, усиливается усилителем 7 и подаётся на вход канала Y осциллографа.
Изменяя частоту генератора и наблюдая за амплитудой сигнала с регистрирующего датчика, можно определить частоту акустического резонанса
в стержне. Наблюдения в режиме X–Y позволяют сравнить сигналы генератора и датчика, а также облегчает поиск резонанса при слабом сигнале.

\section{Методика измерений}
Для определения модуля Юнга материала $E$ используется формула:
\begin{equation}
    u = \sqrt{\frac{E}{\rho}}
\end{equation}
где $u$ -- скорость распространения акустических волн, $\rho$ -- плотность стержня.

Скорость $u$ определяется методом акустического резонанса. При совпадении частоты возбуждения с собственной частотой колебаний стержня $f_n$ возникает стоячая волна с резким увеличением амплитуды. Скорость звука вычисляется по формуле:
\begin{equation}
    u = 2L \frac{f_n}{n}
    \label{eq:15}
\end{equation}
где $L$ -- длина стержня, $n$ -- номер гармоники.

В реальном стержне могут возбуждаться поперечные колебания, поэтому следует учитывать только резонансы, описываемые зависимостью (\ref{eq:15}).

Ширина резонансного максимума определяется добротностью системы:
\[
\Delta f \sim f_{\text{рез}}/Q
\]

Металлические стержни имеют высокую добротность ($Q \sim 10^2 \div10^3$), что приводит к малой ширине резонанса ($\Delta f \sim 1\,\text{Гц}$ при $f \sim 5\,\text{кГц}$) и большому времени установления колебаний ($T_{\text{уст}} \sim Q/f$). Поэтому поиск резонанса проводят, медленно изменяя частоту генератора.

\end{document}
