\documentclass[a4paper, 12pt]{article}
\usepackage[utf8]{inputenc}
\usepackage[russian,english]{babel}
\usepackage[T2A]{fontenc}
\usepackage[left=10mm, top=20mm, right=10mm, bottom=15mm, footskip=13mm]{geometry}
\usepackage{indentfirst}
\usepackage{amsmath,amssymb}
\usepackage{graphicx}
\usepackage[italicdiff]{physics}
\usepackage{float}
\usepackage{array}
\usepackage{physics}
\graphicspath{ {shema/} {graphic/} }
\usepackage{caption}
\captionsetup[figure]{name=Рисунок}
  
\title{Отчет по лабораторной работе 1.2.2

ЭКСПЕРИМЕНТАЛЬНАЯ
ПРОВЕРКА ЗАКОНА
ВРАЩАТЕЛЬНОГО ДВИЖЕНИЯ
НА КРЕСТООБРАЗНОМ
МАЯТНИКЕ}

\author{Максим Осипов, Б03-504}
\date{26.11.2025}

\begin{document}
\maketitle

\section{Аннотация}

\textbf{Цель работы:} 1) экспериментально получить зависимость углового ускорения от момента прикладываемых к маятнику сил, убедиться, что угловое ускорение зависит линейно от момента сил, определить момент инерции маятника; 2) проанализировать влияние сил трения, действующих на ось вращения.

\textbf{В работе используются:} крестообразный маятник, набор перегрузков, секундомер, линейка, штангенциркуль.

\section{Теоретическая справка}

В данной работе экспериментально проверяется уравнение вращательного движения:
\[
I \frac{d\omega}{dt} = M. \tag{1}
\]

Для этого используется крестообразный маятник, устройство которого понятно из рис. 1.

\begin{figure}[h!]
\centering
\includegraphics[width=0.6\textwidth]{установка.png}
\caption{крестообразный маятник}
\label{fig:1}
\end{figure}




Маятник состоит из четырех тонких стержней, укрепленных на втулке под прямым углом друг к другу. Втулка и два шкива различных радиусов (\(r_1\) и \(r_2\)) насажаны на общую ось. Ось закреплена в игольчатых подшипниках, так что вся система может свободно вращаться вокруг горизонтальной оси. Момент инерции маятника можно изменять, передвигая грузы \(m_1\) вдоль стержней.

На один из шкивов маятника навита тонкая нить. Привязанная к ней легкая платформа известной массы служит для размещения перегрузков. Вращающий момент создается силой натяжения нити \(T\):
\[
M_n = rT, \tag{2}
\]
где \(r\) — радиус шкива (\(r_1\) или \(r_2\)). 

Силу \(T\) легко найти из уравнения движения платформы с перегрузком:
\[
mg - T = ma. \tag{3}
\]
Здесь \(m\) — масса платформы с перегрузком.

Если момент сил трения \(M_{тр}\) в подшипниках мал по сравнению с моментом \(M_n\) силы натяжения нити, то из (1), (2) и (3) следует постоянство ускорения \(a\), и, измеряя время \(t\), в течение которого нагруженная платформа из состояния покоя опускается на расстояние \(h\), можно найти ее ускорение \(a\):
\[
a = \frac{2h}{t^2}, \tag{4}
\]
связанное с угловым ускорением \(\beta=d\omega/dt\) простым соотношением:
\[
a=r\frac{d\omega}{dt}=r\beta. \tag{5}
\]

Система уравнений (2) – (5) полностью решает поставленную задачу.

В реальных опытах момент сил трения \(M_{тр}\) может оказаться достаточно большим и существенно исказить результаты опыта. На первый взгляд относительную роль этого момента легко уменьшить, увеличивая массу m. Это, однако, не так, поскольку:
\begin{enumerate}
    \item увеличение массы m ведет к увеличению давления маятника на ось, что вызывает возрастание сил трения;
    \item увеличение m уменьшает время t падения платформы и, следовательно, ухудшает точность измерения времени.
\end{enumerate}

В нашей установке момент сил трения снижен благодаря креплению оси маятника в игольчатых подшипниках (см. рис. 1), однако влияние трения вполне ощутимо и должно приниматься во внимание при обработке результатов опыта.

Для дальнейшей работы удобно преобразовать уравнение (1), выделив момент сил трения в явном виде:
\[
M_n - M_{тр} = I \frac{d\omega}{dt}. \tag{6}
\]

Момент инерции всей системы вычисляется по формуле:
\[
I = I_0 + 4m_1R^2 + 4\frac{m_1l^2}{12} + 4\frac{m_1r^2}{4}, \tag{7}
\]
где \(I_0\) — момент инерции системы без грузов \(m_1\), \(R\) — расстояние центров масс грузов от оси вращения, \(r\) — радиус грузов, \(l\) — образующая цилиндрических грузов.







\end{document}
