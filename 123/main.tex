\documentclass[a4paper, 12pt]{article}
\usepackage[utf8]{inputenc}
\usepackage[russian,english]{babel}
\usepackage[T2A]{fontenc}
\usepackage[left=10mm, top=20mm, right=10mm, bottom=15mm, footskip=13mm]{geometry}
\usepackage{indentfirst}
\usepackage{amsmath,amssymb}
\usepackage{graphicx}
\usepackage[italicdiff]{physics}
\usepackage{float}
\usepackage{array}
\usepackage{physics}
\graphicspath{ {shema/} {graphic/} }
\usepackage{caption}
\captionsetup[figure]{name=Рисунок}
  
\title{Отчет по лабораторной работе 1.2.3

Определение моментов инерции твердых тел с помощью трифилярного подвеса}

\author{Максим Осипов, Б03-504}
\date{19.11.2025}

\begin{document}
\maketitle

\section{Аннотация}

\textbf{Цель работы:} измерение момента инерции ряда тел и сравнение результатов с расчетами по теоретическим формулам; проверка аддитивности моментов инерции и справедливости формулы Гюйгенса-Штейнера.

\textbf{В работе используются:} трифилярный подвес, секундомер, счетчик числа колебаний, набор тел, момент инерции которых надлежит измерить (диск, стержень, полый цилиндр и другие).

\section{Теоретическая справка}

Инерционность при вращении тела относительно оси определяется моментом инерции тела относительно этой оси. Момент инерции твердого тела относительно неподвижной оси вращения вычисляется по формуле:
\[
I = \int r^2 dm,
\]
где \( r \) — расстояние элемента массы тела \( dm \) от оси вращения. Интегрирование проводится по всей массе тела \( m \).

Для экспериментального определения момента инерции используется трифилярный подвес (рис. 1).

\begin{figure}[h!]
\centering
\includegraphics[width=0.8\textwidth]{установка.png}
\caption{Трифилярный подвес}
\label{fig:1}
\end{figure}

При закручивании системы возникает момент сил, возвращающий платформу в положение равновесия. Если пренебречь потерями энергии на трение, уравнение сохранения энергии имеет вид:
\[
\frac{I\dot{\varphi}^2}{2} + mg(z_0 - z) = E,
\]
где \( I \) — момент инерции платформы с телом, \( m \) — масса системы, \( \varphi \) — угол поворота, \( z_0 \) и \( z \) — вертикальные координаты центра платформы в положении равновесия и при повороте соответственно.

Для малых углов \( \varphi \) получаем приближённое выражение:
\[
z \approx z_0 - \frac{Rr\varphi^2}{2z_0}.
\]

Подстановка в уравнение энергии и дифференцирование по времени даёт уравнение крутильных колебаний:
\[
I\ddot{\varphi} + \frac{mgRr}{z_0}\varphi = 0.
\]

Решение этого уравнения описывает гармонические колебания с периодом:
\[
T = 2\pi\sqrt{\frac{Iz_0}{mgRr}}.
\]

Отсюда находим формулу для определения момента инерции:
\[
I = \frac{mgRrT^2}{4\pi^2z_0}.
\]

Учитывая постоянство параметров установки, формулу можно записать в виде:
\[
I = kmT^2,
\]
где \( k = \frac{gRr}{4\pi^2z_0} \) — постоянная установки.

Момент инерции тела определяется как разность моментов инерции платформы с телом и пустой платформы, что следует из аддитивности моментов инерции.

Для обеспечения точности измерений необходимо выполнение условия малости затухания:
\[
\tau \gg T,
\]
где \( \tau \) — время уменьшения амплитуды колебаний в 2–3 раза.










\end{document}
