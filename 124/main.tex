\documentclass[a4paper, 12pt]{article}
\usepackage[utf8]{inputenc}
\usepackage[russian,english]{babel}
\usepackage[T2A]{fontenc}
\usepackage[left=10mm, top=20mm, right=10mm, bottom=15mm, footskip=13mm]{geometry}
\usepackage{indentfirst}
\usepackage{amsmath,amssymb}
\usepackage{graphicx}
\usepackage[italicdiff]{physics}
\usepackage{float}
\usepackage{array}
\usepackage{physics}
\graphicspath{ {shema/} {graphic/} }
\usepackage{caption}
\captionsetup[figure]{name=Рисунок}
  
\title{Отчет по лабораторной работе 1.2.4

Определение главных моментов инерции твердых тел с помощью крутильных колебаний}

\author{Максим Осипов, Б03-504}
\date{19.11.2025}

\begin{document}
\maketitle

\section{Аннотация}

\textbf{Цель работы:} измерить периоды крутильных колебаний рамки при различных положениях закрепленного в ней тела, проверить теоретическую зависимость между периодами крутильных колебаний тела относительно различных осей, определить моменты инерции относительно нескольких осей для каждого тела, по ним найти главные моменты инерции тел и построить эллипсоид инерции.

\textbf{В работе используются:} установка для получения крутильных колебаний (жесткая рамка, имеющая винты для закрепления в ней твердых тел, подвешенная на натянутой вертикально проволоке), набор исследуемых твердых тел, секундомер.


\section{Теоретическая справка}

Инерционные свойства твердого тела при вращении определяются не только массой, но и ее пространственным распределением, которое характеризуется тензором инерции. Тензор инерции представляется симметричной матрицей, приводимой к диагональному виду. Диагональные элементы \(I_x\), \(I_y\), \(I_z\) называются главными моментами инерции тела. 

Геометрическим образом тензора инерции является эллипсоид инерции, уравнение которого в главных осях имеет вид:
\[
I_xx^2 + I_yy^2 + I_zz^2 = 1. \tag{1}
\]
Эллипсоид инерции жестко связан с телом. Если начало координат совпадает с центром масс, эллипсоид называется центральным.

Момент инерции относительно произвольной оси, проходящей через центр эллипсоида, определяется как:
\[
I = \frac{1}{r^2}, \tag{2}
\]
где \(r\) — расстояние до поверхности эллипсоида вдоль этой оси.

\begin{figure}[h]
\centering
\includegraphics[width=0.2\textwidth]{установка.png}
\caption{Установка для крутильных колебаний}
\label{fig:setup}
\end{figure}

В работе используется установка (рис. \ref{fig:setup}), где рамка с закрепленным телом совершает крутильные колебания на проволоке. Уравнение колебаний:
\[
(I+I_p)\frac{d^2\varphi}{dt^2}=-f\varphi. \tag{3}
\]
Период колебаний:
\[
T=2\pi\sqrt{\frac{I+I_p}{f}}. \tag{4}
\]

\begin{figure}[h]
\centering
\includegraphics[width=0.4\textwidth]{параллелепипед.png}
\caption{Оси вращения в параллелепипеде}
\label{fig:axes}
\end{figure}

Для параллелепипеда (рис. \ref{fig:axes}) момент инерции относительно диагонали \(DD'\) выражается через главные моменты:
\[
I_d=I_x\frac{a^2}{d^2}+I_y\frac{b^2}{d^2}+I_z\frac{c^2}{d^2}. \tag{5}
\]

Используя связь (4) между моментом инерции и периодом, получаем:
\[
(a^2+b^2+c^2)T_d^2 = a^2T_x^2+b^2T_y^2+c^2T_z^2. \tag{6}
\]

Аналогично для других осей:
\[
(b^2+c^2)T_E^2 = b^2T_y^2+c^2T_z^2, \tag{7}
\]
\[
(a^2+c^2)T_F^2 = a^2T_x^2+c^2T_z^2, \tag{8}
\]
\[
(a^2+b^2)T_A^2 = a^2T_x^2+b^2T_y^2. \tag{9}
\]
Эти соотношения проверяются экспериментально.


\end{document}
