\documentclass[a4paper]{article}
\usepackage[utf8]{inputenc}
\usepackage[russian,english]{babel}
\usepackage[T2A]{fontenc}
\usepackage[left=10mm, top=20mm, right=18mm, bottom=15mm, footskip=10mm]{geometry}
\usepackage{indentfirst}
\usepackage{amsmath,amssymb}
\usepackage[italicdiff]{physics}
\usepackage{graphicx}
\graphicspath{{images/}}
\DeclareGraphicsExtensions{.pdf,.png,.jpg}
\usepackage{wrapfig}

\usepackage{caption}
\captionsetup[figure]{name=Рисунок}
\captionsetup[table]{name=Таблица}
  
\title{Отчет о выполненой лабораторной работе 1.1.4

Изучение статистических закономерностей на примере измерения фона космического излучения
}

\author{Игнатов Илья, Осипов Максим Б03-604}

\date{\today}


\begin{document}
\maketitle
\newpage


\section{Аннотация}
В данной работе проводится экспериментальное изучение статистических закономерностей счёта частиц радиационного фона. Основное внимание уделено проверке гипотезы о том, что процесс регистрации частиц, являясь независимым и однородным во времени, описывается распределением Пуассона. На большом массиве данных исследуется переход пуассоновского распределения к нормальному. Работа включает расчёт и анализ таких параметров, как среднее значение, дисперсия и стандартное отклонение 



\section{Теоретические сведения}
\subsection{Оборудование}
В работе используются: счетчик Гейгера-Мюллера(СТС-6), блок питания, компьютер с интерфейсом связи со счетчиком, п.
\begin{wrapfigure}{R}{.3\textwidth}
\centering
\includegraphics[width=.2\textwidth]{shema.png}
\caption{Схема включения счетчика}
\end{wrapfigure}

В любой физической лаборатории присутствует естественный радиационный фон, основным источником которого является космическое излучение. Данный фон накладывается на излучение от других источников, если таковые имеются.

Конструкция счётчика Гейгера-Мюллера представляет собой металлический цилиндр, заполненный газом, с двумя электродами: катодом (корпусом счетчика) и анодом (тонкой нитью, натянутой по его оси). На электроды подаётся постоянное напряжение порядка 400 В от блока питания, который часто смонтирован вместе со счётчиком.

Регистрация частиц основана на явлении ударной ионизации. Пролетающие через счётчик космические частицы — в основном, протоны (92\%), альфа-частицы
(6\%) и электроны/позитроны (1\%) ионизируют газовый наполнитель или выбивают электроны из стенок цилиндра. Под действием сильного электрического поля первичные электроны ускоряются, сталкиваются с молекулами газа и порождают лавину вторичных электронов. Этот процесс приводит к возникновению кратковременного импульса тока (газового разряда) в цепи. Данные импульсы регистрируются с помощью компьютерной программы.

Число зарегистрированных частиц зависит от времени измерения, размеров счётчика, от давления и состава газа и от материала, из которого сделаны стенки счётчика.

\subsection{Погрешности}
Наиболее важной характеристикой является среднее число регистрируемых частиц в единицу времени. Если $n_{1}, n_{2}, ..., n_{N}$ - результаты N проведённых в одинаковых условиях измерений, можно вычислить выборочное среднее значение числа измерений:
\[\langle n \rangle = \frac{1}{N}\sum\limits_{i=1}^{N} n_{i}\]
 Согласно закону больших чисел, выборочное среднее стремится к истинному среднему. Если продолжать измерения можно ожидать:
\[\overline{n} = \lim_{N \to \infty} \langle n \rangle\]
Меру флуктуаций среднего значения количественно характеризуют среднеквадратичным отклонением 
$\sigma_{n}$. 
\[
\sigma_n = \sqrt{\frac{1}{N}\sum_{i=1}^{N}(n_i -  \langle n \rangle)^2}
\]
По определению, дисперсия (средний квадрат отклонений) вычисляется как:
\[\sigma_{n}^2 = \frac{1}{N}\sum\limits_{i=1}^{N} (n_{i} - \langle n \rangle)^2 = \langle (n_{i} - \langle n \rangle)^2 \rangle\]

Погрешность среднего значения $\langle n \rangle$ при независимых измерениях связана с погрешностью отдельного измерения формулой:
\[\sigma_{\langle n \rangle} = \frac{\sigma_{n}}{\sqrt{N}}\]
Таким образом, увеличивая количество измерений, среднее значение приближается к «истинному» n. При конечном N истинное среднее с высокой вероятностью лежит в интервале 
\[\overline{n} = \langle n \rangle \pm \frac{\sigma_{n}}{\sqrt{N}}\]

\subsection{Пуассоновский процесс}
Если события однородны во времени и каждое последующее событие не зависит от предыдущих, то такую последовательность событий называют \textit{пуассоновским процессом}.
\newline

Вероятности $\omega_{n}$ обнаружения $n$ частиц в эксперименте для распределения Пуассона задаются формулой:
\[\omega_{n} =  \frac{\overline{n}^{n}}{n!} e^{-\overline{n}}\]
\newline
Для пуассоновского процесса выполняется важное соотношение:
\[\sigma = \sqrt{\overline{n}}\]
То есть среднеквадратичное отклонение равно корню из среднего значения. На практике для выборочных данных можно ожидать выполнение приближённого равенства:
\[\sigma_{n} \approx \sqrt{\langle n \rangle}\]
При больших $\overline{n}$ распределение Пуассона асимптотически приближается к нормальному распределению (распределению Гаусса), которое описывается формулой через $\overline{n}$, $n$ и среднеквадратическое отклонение $\sigma_{n}$:
\[\rho_{n} = \frac{1}{\sqrt{2\pi}\sigma_{n}}e^{-\frac{(n - \overline{n})^2}{2\sigma_{n}^2}}\]
\section{Погрешность эксперимента}



Если подставить основное свойство распределения Пуассона в формулу погрешности среднего значения, то получится среднеквадратичная погрешность определения среднего:
\[\sigma_{\langle n \rangle} = \frac{\sigma_{n}}{\sqrt{N}} = \sqrt{\frac{\langle n \rangle}{N}}\]
Для относительного значения погрешности:
\[\varepsilon_{\langle n \rangle} = \frac{\sigma_{\langle n \rangle}}{\langle n \rangle} = \frac{1}{\sqrt{\langle n \rangle N}}\]

Рассмотрим опыт, в котором интервал измерения t разбит на $N = \frac{t}{\tau}$ промежутков, длительностью $\tau$. В знаменателе полученного выражения, как нетрудно видеть, стоит полное число частиц $N_{0} = \langle n \rangle N = \sum\limits_{i=1}^{N} n_{i}$, зарегистрированных за всё время измерений t. То есть относительная погрешность опыта не зависит от интервалов $\tau$ разбиения серий, и убывает обратно пропорционально корню из общего числа частиц $N_{0}$.

Таким образом, единственный способ увеличить точность опыта — увеличивать общее число регистрируемых частиц за счёт увеличения совокупного времени измерений $\tau$.
\newpage

\section{Обработка результатов}
\subsection{Группировка по \( \tau = 10 \) с}
\begin{figure}[h]
    \centering
    \includegraphics[width=0.8\textwidth]{diagrama_10_sec.png}
    \caption{Распределение вероятностей для  $\tau = 10$}
    \label{fig:my_label}
\end{figure}

На гистограмму наложен граффик нормального распределения. С параметрами среденего $\langle n \rangle$  и среднеквадратичного отклоения $\sigma$

\[
f(x) = \frac{1}{\sigma\sqrt{2\pi}} e^{-\frac{(x - \mu)^2}{2\sigma^2}}
\]
Данные для построение гистограммы 
\begin{table}[h]
\centering
\caption{Данные для построения гистограммы распределения числа срабатываний счетчика за 10 с}
\begin{tabular}{|c|c|c|c|c|c|c|c|c|c|}
\hline
Число импульсов $n_i$ & 1 & 2 & 3 & 4 & 5 & 6 & 7 & 8 & 9 \\ \hline
Число случаев & 2 & 5 & 6 & 12 & 27 & 26 & 53 & 51 & 46 \\ \hline
Доля случаев $w_n$ & 0,005 & 0,012 & 0,015 & 0,03 & 0,068 & 0,065 & 0,132 & 0,128 & 0,115 \\ \hline
\end{tabular}

\vspace{0.5cm}

\begin{tabular}{|c|c|c|c|c|c|c|c|c|c|}
\hline
Число импульсов $n_i$ & 10 & 11 & 12 & 13 & 14 & 15 & 16 & 17 & 18 \\ \hline
Число случаев & 41 & 31 & 32 & 23 & 22 & 6 & 6 & 3 & 5 \\ \hline
Доля случаев $w_n$ & 0,102 & 0,078 & 0,08 & 0,058 & 0,055 & 0,015 & 0,015 & 0,008 & 0,012 \\ \hline
\end{tabular}
\end{table}




\subsection{Группировка по \( \tau = 10 \) с и \( \tau = 40 \) }

\begin{figure}[h]
    \centering
    \includegraphics[width=0.8\textwidth]{diagrama_10_40_sec.png}
    \caption{Распределение вероятностей для  $\tau = 10$ и $\tau = 40$ c}
    \label{fig:my_label}
\end{figure}


Для обоих измерений вычислим среденее значение $\langle n \rangle$, среднеквадратичное отклоение отдельного эсперимета и погрешность среднего

\begin{equation*}
\langle n \rangle = \frac{1}{N}\sum_{i=1}^{N} n_{i}
\quad
\sigma_n = \sqrt{\frac{1}{N}\sum_{i=1}^{N}(n_i - \langle n \rangle)^2}
\quad
\sigma_{\langle n \rangle} = \frac{\sigma_n}{\sqrt{N}}
\end{equation*}

Также убедимся в справедливости формулы $\sigma_{n} \approx \sqrt{\langle n \rangle}$  

$t = 10 \: c$: $\overline n_1 = 10,09, \:\sigma_{1} = 3,17 ,\: \sigma_{\overline n_1} = 0,05$; $3.11 \approx \sqrt{10.09} = 3.11$.
		\par
		$t = 40 \: c$: $\overline n_2 = 40,34, \:\sigma_{2} = 6,35, \: \sigma_{\overline n_2} = 0,1$; $6,55 \approx \sqrt{40,34} = 6,55$.
		\item Найдём процент случаев, когда отклонение от среднего не превышает $\sigma, 2\sigma$. Сравним результаты с теоретическими оценками.
		\begin{center}
			\begin{tabular}{|c|c|c|}
				\hline
				Ошибка & Доля случаев, \% & Теоретическая оценка \\\hline
				$\sigma_1$ & 75,25 & 68\\\hline
				$2\sigma_1$ & 96,5 & 95\\\hline \hline
				$\sigma_2$ & 72 & 68\\\hline
				$2\sigma_2$ & 95 & 95\\\hline 
			\end{tabular}    
		\end{center}
		\item  Найдём относительную погрешность средних значений:
		\[ \varepsilon_1 = \frac{1}{\sqrt{\overline{n_1}N_1}} = 0,49 \%, \:\:\: \varepsilon_2 = \frac{1}{\sqrt{\overline{n_2}N_2}} = 0,24 
        \%.\]

        \begin{table}[h!]
\centering
\large
\caption{Число срабатываний за 20 с}
\begin{tabular}{|c|*{10}{c|}}
\hline
\textbf{№опыта} & \textbf{1} & \textbf{2} & \textbf{3} & \textbf{4} & \textbf{5} & \textbf{6} & \textbf{7} & \textbf{8} & \textbf{9} & \textbf{10} \\
\hline
0               & 26         & 26         & 26         & 22         & 28         & 18         & 19         & 28         & 17         & 27          \\
\hline
10              & 22         & 26         & 22         & 20         & 25         & 23         & 32         & 21         & 24         & 30          \\
\hline
20              & 22         & 24         & 26         & 18         & 20         & 19         & 23         & 39         & 28         & 21          \\
\hline
30              & 28         & 27         & 19         & 20         & 25         & 29         & 31         & 36         & 30         & 22          \\
\hline
40              & 19         & 35         & 29         & 20         & 30         & 15         & 16         & 20         & 22         & 15          \\
\hline
50              & 24         & 21         & 17         & 18         & 23         & 24         & 21         & 19         & 37         & 17          \\
\hline
60              & 23         & 27         & 12         & 20         & 29         & 31         & 24         & 24         & 22         & 25          \\
\hline
70              & 20         & 26         & 19         & 27         & 26         & 30         & 24         & 25         & 24         & 18          \\
\hline
80              & 24         & 23         & 27         & 32         & 26         & 25         & 30         & 25         & 22         & 22          \\
\hline
90              & 25         & 20         & 18         & 26         & 26         & 27         & 28         & 22         & 22         & 33          \\
\hline
100             & 14         & 31         & 26         & 32         & 20         & 25         & 27         & 24         & 24         & 28          \\
\hline
110             & 28         & 18         & 30         & 23         & 35         & 28         & 21         & 25         & 28         & 27          \\
\hline
120             & 27         & 29         & 23         & 24         & 20         & 14         & 23         & 25         & 27         & 34          \\
\hline
130             & 22         & 26         & 20         & 20         & 16         & 27         & 19         & 28         & 26         & 24          \\
\hline
140             & 22         & 22         & 25         & 22         & 24         & 22         & 32         & 32         & 23         & 31          \\
\hline
150             & 17         & 28         & 25         & 21         & 27         & 24         & 24         & 27         & 22         & 24          \\
\hline
160             & 21         & 25         & 30         & 16         & 34         & 36         & 34         & 25         & 24         & 32          \\
\hline
170             & 22         & 17         & 18         & 20         & 28         & 24         & 24         & 40         & 25         & 16          \\
\hline
180             & 27         & 21         & 26         & 22         & 19         & 25         & 28         & 20         & 30         & 29          \\
\hline
190             & 27         & 17         & 27         & 21         & 26         & 24         & 18         & 32         & 30         & 23          \\
\hline
\end{tabular}
\end{table}








\begin{table}[h!]
\centering
\large
\caption{Число срабатываний счетчика за 40 с}
\label{tab:experiments}
\begin{tabular}{|c|*{10}{c|}} % 11 колонок, все с вертикальными линиями
\hline % Верхняя горизонтальная линия
\textbf{№ опыта} & \textbf{1} & \textbf{2} & \textbf{3} & \textbf{4} & \textbf{5} & \textbf{6} & \textbf{7} & \textbf{8} & \textbf{9} & \textbf{10} \\
\hline % Линия после заголовка
0                & 52         & 48         & 46         & 47         & 44         & 48         & 42         & 48         & 53         & 54          \\
\hline
10               & 46         & 44         & 39         & 62         & 49         & 55         & 39         & 54         & 67         & 52          \\
\hline
20               & 54         & 49         & 45         & 36         & 37         & 45         & 35         & 47         & 40         & 54          \\
\hline
30               & 50         & 32         & 60         & 48         & 47         & 46         & 46         & 56         & 49         & 42          \\
\hline
40               & 47         & 59         & 51         & 55         & 44         & 45         & 44         & 53         & 50         & 55          \\
\hline
50               & 45         & 58         & 45         & 51         & 52         & 46         & 53         & 63         & 46         & 55          \\
\hline
60               & 56         & 47         & 34         & 48         & 61         & 48         & 40         & 43         & 47         & 50          \\
\hline
70               & 44         & 47         & 46         & 64         & 54         & 45         & 46         & 51         & 51         & 46          \\
\hline
80               & 46         & 46         & 70         & 59         & 56         & 39         & 38         & 52         & 64         & 41          \\
\hline
90               & 48         & 48         & 44         & 48         & 59         & 44         & 48         & 50         & 50         & 53          \\
\hline
\end{tabular}
\end{table}
Данные для построения диаграммы разбиение по 40 секунд 
\begin{table}[h]
\caption{Данные для построения гистограммы распределения числа срабатываний счетчика за 40 с}
\centering
\begin{tabular}{|c|c|c|c|c|c|c|c|c|c|c|c|c|}
\hline
Число импульсов & 26 & 27 & 28 & 29 & 31 & 32 & 33 & 34 & 35 & 36 & 37 & 38 \\ \hline
Число случаев   & 1  & 2  & 1  & 1  & 3  & 3  & 4  & 3  & 1  & 8  & 8  & 4  \\ \hline
Доля случаев    & 0,01 & 0,02 & 0,01 & 0,01 & 0,03 & 0,03 & 0,04 & 0,03 & 0,01 & 0,08 & 0,08 & 0,04 \\ \hline
\end{tabular}

\vspace{0.5cm}

\begin{tabular}{|c|c|c|c|c|c|c|c|c|c|c|c|c|}
\hline
Число импульсов & 39 & 40 & 41 & 42 & 43 & 44 & 45 & 46 & 47 & 48 & 49 & 50 \\ \hline
Число случаев   & 8  & 6  & 4  & 5  & 7  & 7  & 5  & 6  & 3  & 1  & 2  & 3  \\ \hline
Доля случаев    & 0,08 & 0,06 & 0,04 & 0,05 & 0,07 & 0,07 & 0,05 & 0,06 & 0,03 & 0,01 & 0,02 & 0,03 \\ \hline
\end{tabular}

\vspace{0.5cm}

\begin{tabular}{|c|c|}
\hline
Число импульсов & 53 \\ \hline
Число случаев   & 2  \\ \hline
Доля случаев    & 0,02 \\ \hline
\end{tabular}
\end{table}



\newpage
\section{Вывод}
В ходе работы были получены данные интенсивности радиационного фона. С помощью методов оценки погрешностей и теории вероятности мы нашли средние значения для разбиений по 10с и 40с.
 
 $\overline n_1 = 10.09 \pm 0,05$ и    $\overline n_2 = 40.34 \pm 0,1$. Относительные погрешности определения $n_1$ и $n_2$ совпадают и весьма невелики ($0.49 \%;  0,24\%$). Проверено, что результаты измерений соответствуют характерному для распределения Пуассона равенству: $\sigma = \sqrt{n_0}$. При большом числе регистраций частиц выполняются свойства нормального распределения, так-же гистограммы сходятся с теоретическими графиками нормального распределения

\end{document}
	

    












\end{document}