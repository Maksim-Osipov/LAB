\documentclass[a4paper, 12pt]{article}
\usepackage[utf8]{inputenc}
\usepackage[russian]{babel}
\usepackage[T2A]{fontenc}
\usepackage[left=10mm, top=20mm, right=10mm, bottom=15mm]{geometry}
\usepackage{indentfirst}
\usepackage{amsmath}
\usepackage{graphicx}
\usepackage{float}
\usepackage{caption}
\graphicspath{ {graphic/} }
\captionsetup[figure]{name=Рисунок}
  
\title{Отчет по лабораторной работе 2.1.5 \\ Исследование термических эффектов при упругих деформациях}
\author{Фамилия Имя, Группа}
\date{\today}

\begin{document}
\maketitle

\section*{Аннотация}
\textbf{Цель работы:} 1) исследовать зависимость удлинения резины от нагрузки при постоянной температуре; 2) измерить изменение температуры резины при адиабатическом растяжении и определить её теплоёмкость.

\textbf{Оборудование:} резиновая полоса в теплоизолированном кожухе, набор грузов, термопара (медь-константан), усилитель, осциллограф/микровольтметр, измерительная линейка.

\section{Теоретическая справка}

Рассматривается растяжение тонкой полосы длиной $l$ под действием силы $f$. Работа, совершаемая над образцом:
\[
\delta A = -f dl + P dV,
\]
где $P$ — атмосферное давление. Для резины вторым слагаемым можно пренебречь, так как $P dV \ll f dl$. Тогда первое начало термодинамики принимает вид:
\[
dU = T dS + f dl. \qquad (1)
\]

Вводится свободная энергия $F = U - TS$, дифференциал которой с учётом (1) равен:
\[
dF = -S dT + f dl. \qquad (2)
\]
Отсюда следуют соотношения:
\[
f = \left( \frac{\partial F}{\partial l} \right)_T, \quad S = -\left( \frac{\partial F}{\partial T} \right)_l. \qquad (3)
\]

Дифференцируя первое соотношение (3) по $T$, а второе по $l$, и учитывая равенство смешанных производных, получаем соотношение Максвелла:
\[
\left( \frac{\partial f}{\partial T} \right)_l = -\left( \frac{\partial S}{\partial l} \right)_T. \qquad (4)
\]

Теплота, поглощаемая образцом при изотермическом растяжении, выражается как:
\[
\delta Q|_{T} = T dS|_{T} = -T \left( \frac{\partial f}{\partial T} \right)_{l} dl|_{T}. \qquad (5)
\]

Из (1) можно получить общее выражение для силы:
\[
f = \left( \frac{\partial U}{\partial l} \right)_T - T \left( \frac{\partial S}{\partial l} \right)_T. \qquad (6)
\]
В обычных твёрдых телах доминирует первое слагаемое (изменение внутренней энергии), а в резине — второе (энтропийный вклад).

Большинство материалов при адиабатическом растяжении охлаждаются. Резина проявляет аномальный эффект: при быстром растяжении она нагревается.

\subsection{Термодинамика резины}
Основная модель — \textbf{«идеальная резина»}, в которой внутренняя энергия зависит только от температуры: $U = U(T)$. Упругость обусловлена изменением энтропии $S$.

Для идеальной резины при $T = const$ ($dU=0$) из (6) следует:
\[
f = -T \left( \frac{\partial S}{\partial l} \right)_T. \qquad (7)
\]
Сила растяжения пропорциональна абсолютной температуре: $f(T, l) = (T/T_0) \cdot \tilde{f}(l/l_0)$.

Из соотношения Максвелла (4) и выражения (7) получаем:
\[
f = T \left( \frac{\partial f}{\partial T} \right)_l. \qquad (8)
\]

При \textbf{адиабатическом растяжении} ($dS=0$) из (1) имеем $dU = f dl$. Так как для идеальной резины $dU = C_l dT$, где $C_l$ — теплоёмкость при постоянной длине, то для малых $\Delta T$:
\[
\Delta T \approx \frac{1}{C_l} \int_{l_0}^{l} f dl = \frac{A_{\text{внеш}}}{C_l}. \qquad (9)
\]

В общем случае для любого тела изменение температуры при адиабатическом растяжении выражается как:
\[
\left( \frac{\partial T}{\partial l} \right)_S = \frac{T}{C_l} \left( \frac{\partial f}{\partial T} \right)_l. \qquad (10)
\]
Знак эффекта определяется знаком коэффициента теплового расширения $\alpha = \frac{1}{l} \left( \frac{\partial l}{\partial T} \right)_f$. Для большинства тел $\alpha > 0$, и они охлаждаются; для резины при достаточно больших растяжениях $\alpha < 0$, что приводит к нагреву.

\subsection{Закон растяжения резины}
Эмпирическая зависимость (модель Кюна для полимерной сетки):
\[
f(T, \lambda) = \frac{s_0 E}{3} \left( \lambda - \frac{1}{\lambda^2} \right), \qquad (11)
\]
где $\lambda = l/l_0$ — относительное удлинение, $s_0$ — площадь сечения, $E \sim T$ — модуль Юнга. При малых деформациях ($\lambda \to 1$) переходит в закон Гука: $f \approx s_0 E (\lambda - 1)$.

Для модели идеальной резины изменение энтропии как функции растяжения при постоянной температуре даётся выражением:
\[
\Delta S(\lambda) \approx -\text{const} \cdot \left( \lambda^2 + \frac{2}{\lambda} \right). \qquad (12)
\]

\subsection{Молекулярная структура}
Резина представляет собой сетку из полимерных цепей, соединённых поперечными «сшивками». Растяжение вызывает разворачивание свёрнутых макромолекул, что увеличивает порядок и \textbf{уменьшает энтропию}. Это объясняет энтропийную природу упругости и нагрев при адиабатическом растяжении.

\section{Экспериментальная установка}
Установка (рис. \ref{fig:setup}) включает:
\begin{enumerate}
    \item \textbf{Образец (1)} — резиновая полоса.
    \item \textbf{Теплоизолирующий кожух (2)} для минимизации теплообмена.
    \item \textbf{Зажимы (3)} для крепления. Нижний зажим перемещается по направляющим (4).
    \item \textbf{Линейка (5)} для измерения удлинения $\Delta l$.
    \item \textbf{Платформа (6)} для размещения грузов, создающих силу $F = mg$.
    \item \textbf{Термопара (9)} (медь-константан). Рабочий спай вшит в образец, компенсационный (10) находится в кожухе. Подключена к усилителю и осциллографу/микровольтметру.
    \item \textbf{Упор (7)} для фиксации максимального удлинения.
\end{enumerate}

\begin{figure}[H]
\centering
\includegraphics[width=0.6\linewidth]{установка.png}
\caption{Схема экспериментальной установки.}
\label{fig:setup}
\end{figure}

\end{document}